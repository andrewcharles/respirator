
\section{Building the Software}
SPAC is easy to build on a variety of platforms using the supplied makefile. It
has no external dependencies and is written entirely in FORTRAN 90.  SPAC has
been tested on several flavours of Linux and Mac OS with the Gfortran, Intel
compilers and the Portland group fortran compiler. There may be issues building
with the IBM XLF compiler.

Download the source code from the subversion server (* these details have changed *)

\begin{verbatim}
svn co http://svn.assembla.com/svn/sp_vdw
\end{verbatim}

or obtain a tar archive of the source by some other means. The tar file will
unzip into its own directory. Unpack the tar file 

\begin{verbatim}
tar -xvf spac.tar.gz
\end{verbatim}

Check the makefile, and set compiler names and flags appropriate to your
system. Then run make

\begin{verbatim}
make
\end{verbatim}

and the code should be ready to go. Running the executable \texttt{fsph} in the
build directory will use the local copy of the configuration file
\texttt{spinput.txt} and write output in ASCII format to the local directory.
The python script \texttt{run.py} provides higher level control over program
execution. It is highly recommended that you avail yourself of the assocated
python based visualisation and analysis package \texttt{VSP}.

\section{Running the Model}
Several sample configuration files are included and are used as the basis for
the automated tests.

\subsection{run.py}
\texttt{run.py} is a python program that leverages the \texttt{os} and \texttt{shutil} modules
to manage the creation of directories, staging of output files and generation of plots for
smooth particle runs.

Create a directory in \texttt{../runs} called \texttt{inital\_run} and use the default
config spinput.txt:

\begin{verbatim}
run.py -n inital_run
\end{verbatim}

Create a new directory called \texttt{followup\_run}, using the current version of spinput.txt
and the final state of the previous run as the initial state:

\begin{verbatim}
run.py -c inital_run -n followup_run -a
\end{verbatim}

Use a different directory to the defaut \texttt{../runs}:

\begin{verbatim}
run.py -n new_run -b other_directory
\end{verbatim}

Run the first integrated test:

\begin{verbatim}
run.py --test 1
\end{verbatim}

Run all integrated tests:

\begin{verbatim}
testall.py
\end{verbatim}



